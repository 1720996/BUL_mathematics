\documentclass{article}
\usepackage[utf8]{inputenc}
\usepackage{url,amsmath,graphicx,amssymb,booktabs}
\usepackage[top=1.5cm, bottom=1.5cm, left=2.5cm, right=2.5cm]{geometry}

\title{MA3614 - Complex Variable Methods and Applications}
\author{Luke Dando}
\date{\today}

\begin{document}

\maketitle

\tableofcontents

\section{Introductory Material}
\subsection{Euler's Formula}
\begin{equation}
    r(e^{ix})= r(\cos(x)+i\sin(x)).
\end{equation}

\subsection{Polar Form of a `Complex' Number Raised to a Power}
First, convert the real or imaginary base to polar form, then divide the power by the original exponent and add the unit circle in equal sized portions. Here's a general worked solution where the base has already been converted to its polar form:
\begin{align}
    R=z^n&=r^n(e^{i(\theta+2k\pi)}),\quad \quad k = 1,2,\ldots,n\\
    z&=r(e^{i(\frac{\theta+2k\pi}{n})}),\\
    &=re^{i(\frac{\theta}{n}+\frac{2k\pi}{n})}.
\end{align}

\subsection{Argument}
\begin{equation}
    \text{Arg}(z)=\text{Arg}(re^{i\theta})=\theta.
\end{equation}

\subsection{Real and Imaginary Parts of a Complex Number}
Let $z=a+ib$
\begin{align}
    \text{Re}(z) &= a,\\
    \text{Im}(z) &= b.
\end{align}

\section{Functions of a Complex Variable}
\subsection{Geometric Series}
\begin{equation}
    1+z+z^2+\cdots+z^n = \frac{z^{n+1}-1}{z-1}.
\end{equation}


\section{The Complex Derivative and Analytic Functions}
\subsection{Differentials}
\begin{align}
    &\frac{\partial}{\partial x}(\sin(\alpha x)) = \alpha\cos(\alpha x).\\
    &\frac{\partial}{\partial x}(\cos(\alpha x)) = -\alpha\sin(\alpha x).\\
    &\frac{\partial}{\partial x}(\cosh(\alpha x)) = \alpha\sinh(\alpha x).\\
    &\frac{\partial}{\partial x}(\sinh(\alpha x)) = \alpha\cosh(\alpha x).
\end{align}

\subsection{Integrals}
\begin{align}
    &\int\sin(\alpha x)\, dx =  -\frac{1}{\alpha}\cos(\alpha x) + C.\\
    &\int\cos(\alpha x)\, dx =  \frac{1}{\alpha}\sin(\alpha x) + C.\\
    &\int\cosh(\alpha x)\, dx =  \frac{1}{\alpha}\sinh(\alpha x) + C.\\
    &\int\sinh(\alpha x)\, dx =  \frac{1}{\alpha}\cosh(\alpha x) + C.
\end{align}

\subsection{Analytical Functions}
\subsubsection{Cauchy-Riemann Equations}
A function (generally given as $f = u +iv$) is \textbf{\textit{analytic}} on the complex plane if both the Cauchy-Riemann equations~(\ref{eq:CR}) are satisfied.
\begin{equation}
    \frac{\partial u}{\partial x} = \frac{\partial v}{\partial y},\qquad \frac{\partial v}{\partial x} = -\frac{\partial u}{\partial y}\label{eq:CR}
\end{equation}

\subsection{Harmonic Functions}
\subsubsection{Harmonic Equations}
An equation $u(x,y)$ is considered \textbf{\textit{harmonic}} if:
\begin{equation}
    \nabla^2u = \frac{\partial^2u}{\partial x^2}+\frac{\partial^2u}{\partial y^2} =0.
\end{equation}

\subsection{Exponential forms of Trigonometric Functions}
\begin{align}
    &\cos(x)=\frac{e^{ix}+e^{-ix}}{2}, && -1\leq\cos(x)\leq 1.\\
    &\sin(x)=\frac{e^{ix}-e^{-ix}}{2}, && -1\leq\sin(x)\leq 1.\\
    &\cosh(x)=\frac{e^{x}+e^{-x}}{2}, && 1\leq \cosh(x)<\infty.\\
    &\sinh(x)=\frac{e^{x}-e^{-x}}{2}, && -\infty<\sinh(x)<\infty.
\end{align}

\section{The Elementary Functions of a Complex Variable}
\section{Contour Integrals}
\section{Other Useful Information}

\subsection{De Moivre's Formula}
Not sure if taught, but could be useful in some contexts:
\begin{equation}
    (\cos(x) + i\sin(x))^n = \cos(nx) + i\sin(nx).
\end{equation}

Here's proof if you have time to add it to ensure marks:
\begin{align}
    (\cos(x) + i\sin(x))^n &= (e^{ix})^n,\\
    &=e^{inx},\\
    &=\cos(nx) + i\sin(nx).
\end{align}

\end{document}
