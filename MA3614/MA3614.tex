\documentclass{article}
\usepackage[utf8]{inputenc}
\usepackage{url,amsmath,graphicx,amssymb,booktabs}
\usepackage[top=1.5cm, bottom=1.5cm, left=2.5cm, right=2.5cm]{geometry}

\title{MA3614 - Complex Variable Methods and Applications}
\author{Luke Dando}
\date{\today}

\begin{document}

\maketitle

\tableofcontents

\section{Cauchy's Integral Formula, Consequences and Bounds}
\subsection{The Cauchy Integral Formula}
The basic formula is as follows:
\begin{equation}
    f(z)=\frac{1}{2\pi i}\oint_\Gamma \frac{f(\zeta)}{\zeta-z}\,d\zeta
\end{equation}
This can be generalised to give
\begin{equation}
    f^{(n)}(z)=\frac{n!}{2\pi i}\oint_\Gamma \frac{f(\zeta)}{(\zeta-z)^{n+1}}\,d\zeta
,\quad n=0,1,2,\ldots \end{equation}
\subsection{Morera's Theorem}
If
\begin{equation}
    \oint_\Gamma f(z)\,dz=0
\end{equation}
for all loops $\Gamma$, then $f$ is analytic inside $\Gamma$

\subsection{Harmonic Functions}
Let $f$ be analytic. Then all partial derivatives of $u$ and $v$ also exist, such that. 
\begin{equation}
    f^\prime(z)=\frac{\partial u}{\partial x} + i \frac{\partial v}{\partial x} = \frac{\partial v}{\partial y} -i\frac{\partial u}{\partial y}.
\end{equation}
where $u$ and $v$ are harmonic.

\end{document}
